\documentclass{article}
\usepackage{amsmath}
\usepackage{graphicx}

\title{Recursive Motion Equations for Radiation Coordinate Systems within The Unified Theory of Energy}
\author{Unified Theory of Energy Framework}
\date{\today}

\begin{document}

\maketitle

\section{Introduction}
Traditional physics relies on a synthetic Cartesian framework where motion is measured as displacement over time. This document introduces a recursive motion framework where all motion is relative to a dominant Radiation Source, and absolute velocity is replaced by cyclic motion expressed in radians and frequency.

\section{Motion as a Function of Frequency}
In the standard Cartesian view:
\begin{equation}
    v = \frac{dx}{dt}
\end{equation}
where $v$ is velocity, $x$ is displacement, and $t$ is time.

Acceleration is defined as:
\begin{equation}
    a = \frac{dv}{dt}
\end{equation}

In the recursive framework, motion is always cyclic within a Radiation Coordinate System, and the fundamental measure of motion is angular frequency:
\begin{equation}
    \omega = \frac{2\pi}{T}
\end{equation}
where $T$ is the orbital period of the Particle within its Radiation Source.

Velocity is defined relative to the Radiation Source:
\begin{equation}
    v = r \cdot \omega
\end{equation}

Acceleration is:
\begin{equation}
    a = r \cdot \omega^2
\end{equation}

\section{Generalized Recursive Motion Law}
Since motion only exists within a Radiation Source’s coordinate system, any Particle within a system follows:
\begin{equation}
    \omega_n = \frac{\omega_{n-1}}{r_n}
\end{equation}
where:
\begin{itemize}
    \item $\omega_n$ is the angular frequency within the current coordinate system,
    \item $\omega_{n-1}$ is the frequency of the parent Radiation Source,
    \item $r_n$ is the distance of the Particle from its Radiation Source.
\end{itemize}

Each nested system inherits motion from its parent but is scaled by its relative position.

\section{Velocity Within Radiation Coordinate Systems}
Velocity is recursively defined as:
\begin{equation}
    v_n = r_n \cdot \omega_n
\end{equation}
\begin{equation}
    v_n = \frac{r_n}{r_{n-1}} v_{n-1}
\end{equation}
where:
\begin{itemize}
    \item $v_n$ is the observed velocity at level $n$,
    \item $v_{n-1}$ is the velocity at the parent level.
\end{itemize}

This means motion is scale-dependent, and an object’s velocity shifts as it moves across Radiation Coordinate Systems.

\section{Energy and Momentum in Radiation Coordinate Systems}
Since motion is cyclic, kinetic energy is:
\begin{equation}
    E_k = \frac{1}{2} m (r \omega)^2
\end{equation}

Momentum is given by:
\begin{equation}
    p = m r \omega
\end{equation}

These equations replace traditional linear energy and momentum equations, meaning all energy is inherently rotational at the correct scale.

\section{Gravitation as a Recursive Motion Effect}
Instead of treating gravity as a Cartesian force, we recognize it as an effect of recursive motion:
\begin{equation}
    g_n = r_n \omega_n^2
\end{equation}

This explains why gravitational acceleration weakens with distance: it is a function of recursive frequency interactions rather than absolute force.

\section{Implications}
\begin{itemize}
    \item There is no need for absolute space—motion is only meaningful within the Radiation Source's coordinate system.
    \item The speed of light becomes a function of the dominant Radiation Source, not an absolute universal constant.
    \item Gravitation emerges from nested rotational interactions, meaning gravity is a recursive motion phenomenon rather than a standalone force.
\end{itemize}

\section{Conclusion}
This framework redefines motion as cyclic and relative within Radiation Coordinate Systems. By abandoning the Cartesian assumption of absolute space, this approach provides a unified view of planetary motion, gravity, and energy interactions across all scales.

\end{document}
